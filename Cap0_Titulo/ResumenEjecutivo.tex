\chapter*{Resumen Ejecutivo}
En barraquilla, el sector metal-mecánico encargado de operaciones de mecanizado como el fresado y torneado con máquinas CNC es limitado, lo que se traduce en un pobre desarrollo en la industria metal-mecánica. El manejo limitado de la maquinaria CNC para mecanizado se debe al elevado costo de los equipos que se requieren para el mecanizado de piezas con geometría compleja y altas tolerancias de fabricación. En el departamento del atlántico según \cite{lora2012determinantes} tiene un 4\% de participación en la industria metal-mecánica. Viendo el porcentaje de la industria en el atlántica se observa una oportunidad de negocio, donde se plantea la idea de diseñar una maquina CNC para fresado de tres ejes con la cual se trabajarían aluminios y aceros dúctiles con un bajos costo de inversión y con una disminución en consumo de energía durante la operación. Con este fin los estudiantes de pregrado de la universidad del norte se han propuesto a diseñar una maquina herramienta de tres ejes para mecanizado.

El presente trabajo cubre un marco teórico donde se describen los procesos de mecanizado, las fases de diseño conceptual, diseño básico y diseño de detalle de una maquina herramienta CNC para fresado, enfocado principalmente al direccionamiento de los componentes de la arquitectura del mecanismo que siguen las trayectorias de la operación, siguiendo la metodología \cite{dieter2012engineering}. El proyecto comenzó con el planteamiento del problema, una revisión del estado del arte, de la técnica y una revisión de patentes; donde se encontraron opciones de arquitecturas y puntos donde se puede mejorar algo ya existente, la definición de requerimientos y especificaciones utilizando el método de despliegue de la función de la calidad (QFD). Posteriormente se realizó una descomposición funcional donde se identificaron las funciones de primer, segundo y tercer nivel, teniendo así tres alternativas de la solución teniendo en cuentas diferentes concetos de la solución para cada función. Las tres alternativas generadas se evaluaron con el método AHP. Por último, se realizó el diseño básico y detallado de la alternativa seleccionada.

Como resultado del proyecto realizado se determinó el uso como mecanismo de actuación una arquitectura de robot paralelo RRPRR basado en \cite{petko2005mechatronic} con la cual se busca una disminución de los costos de consumo energético, una precisión competitiva y un costo de adquisición menor en comparación maquinas CNC con similar espacio de trabajo que se encuentran en el mercado actual. Además, se lograron tener dimensiones optimas del mecanismo con el uso de algoritmos genéticos.
